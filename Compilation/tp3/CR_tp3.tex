\documentclass{article}
\usepackage[french]{babel}
\usepackage[utf8]{inputenc}
\usepackage{listings} % Required for insertion of code
\usepackage{courier} % Required for the courier font

\usepackage[top=1in]{geometry}
\usepackage{textcomp}

\usepackage{xcolor}
\usepackage{filecontents}

% Margins
\topmargin=-0.65in
\evensidemargin=0in
\oddsidemargin=-0.15in
\textwidth=6.5in
\textheight=9.0in
\headsep=0.25in
\headheight=30pt
\linespread{1} % Line spacing
\marginparwidth=0.05in

\setlength\parindent{0pt} % Removes all indentation from paragraphs



%----------------------------------------------------------------------------------------
%	NAME AND CLASS SECTION
%----------------------------------------------------------------------------------------

\newcommand{\hmwkTitle}{} % Assignment title
\newcommand{\hmwkDueDate}{2 décembre 2014} % Due date
\newcommand{\hmwkClass}{Compilation : TP3} % Course/class
\newcommand{\hmwkClassTime}{} % Class/lecture time
\newcommand{\hmwkClassInstructor}{} % Teacher/lecturer
\newcommand{\hmwkAuthorName}{Valentin { \sc Esmieu} \\ Corentin {\sc Nicole} \\ \textit{groupe 1.2}} % Your name

%----------------------------------------------------------------------------------------
%	TITLE PAGE
%----------------------------------------------------------------------------------------

\title{
\vspace{2in}
\textmd{\textbf{\hmwkClass}}\\
\normalsize\vspace{0.1in}\small{\hmwkDueDate}\\
\vspace{0.1in}\large{\textit{\hmwkClassInstructor\ \hmwkClassTime}}
\vspace{3in}
}

\author{\hmwkAuthorName}
\date{} % Insert date here if you want it to appear below your name


\begin{document}


\maketitle
\newpage

\section{Code}
\lstinputlisting[
  caption    = {{\bf main.ml}},
 % breaklines,
]{main.ml}


\begin{tabular}{ccccccccccccccccccccccccccccccccccccccccccccccc}
   	& & & & & & & & & & & & & & & & & & & & & & & & & & & & & & & & & & & & & & & & & & & \\
	& & & & & & & & & & & & & & & & & & & & & & & & & & & & & & & & & & & & & & & & & & & \\
  	\hline
	& & & & & & & & & & & & & & & & & & & & & & & & & & & & & & & & & & & & & & & & & & & \\
	& & & & & & & & & & & & & & & & & & & & & & & & & & & & & & & & & & & & & & & & & & & \\
\end{tabular}

\lstinputlisting[
  caption    = {{\bf arbre.ml}},
 % breaklines,
]{arbre.ml}

\newpage

\lstinputlisting[
  caption    = {{\bf lexer.mll}},
 % breaklines,
]{lexer.mll}

\begin{tabular}{ccccccccccccccccccccccccccccccccccccccccccccccc}
   	& & & & & & & & & & & & & & & & & & & & & & & & & & & & & & & & & & & & & & & & & & & \\
	& & & & & & & & & & & & & & & & & & & & & & & & & & & & & & & & & & & & & & & & & & & \\
  	\hline
	& & & & & & & & & & & & & & & & & & & & & & & & & & & & & & & & & & & & & & & & & & & \\
	& & & & & & & & & & & & & & & & & & & & & & & & & & & & & & & & & & & & & & & & & & & \\
\end{tabular}

\lstinputlisting[
  caption    = {{\bf parser.mly}},
 % breaklines,
]{parser.mly}


\newpage
\section{Questions}
\paragraph{Question 1}
\emph{Quel est l'intérêt d'utiliser un crible avec ocamllex ?}

Ceci permet de filtrer l'ensemble des lexèmes pour génerer les Ident correspondants. Dans ce TP, pour limiter le nombre de transitions, nous utilisons une table pour stocker les Ident particuliers tels que \emph{begin}, \emph{end}, etc.


\paragraph{Question 2}
\emph{Quelle est la différence, d'un point de vue pratique, entre écrire une grammaire sous forme LL et LR ?}

Dans une grammaire LR, la détection d'erreur se fait plus tôt et permet une couverture plus large. De plus, une grammaire LR permet de traiter des grammaires ambiguës.


\paragraph{Question 3}
\emph{Que signifie une colonne vide dans une table SLR ?}

Une colonne vide indique que le token correspondant n'est jamais utilisé.







\end{document}
